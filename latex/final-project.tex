\documentclass{article}
\usepackage{graphicx} % Required for inserting images
\usepackage{enumitem}
\usepackage{geometry}
\usepackage{tikz}
\usepackage{amssymb}
\usepackage{pifont}
\usepackage{amsmath}
\usepackage{array}
\usetikzlibrary{automata, positioning, arrows}
\tikzset{
->, % makes the edges directed
>=stealth, % makes the arrow heads bold
node distance=3cm, % specifies the minimum distance between two nodes. Change if necessary.
every state/.style={thick, fill=gray!10}, % sets the properties for each ’state’ node
initial text=$ $, % sets the text that appears on the start arrow
}

\geometry{
    a4paper,
    total={7in, 8in},
    top=1in, 
    bottom=1in,
}
\usepackage{titling}
\setlength{\droptitle}{-2em}  

\title{EECS 510 Final Project}
\author{Andrew Huang and John Rader}
\date{May 2025}
\begin{document}

\maketitle

\newlist{mylist}{enumerate}{1}
\setlist[mylist, 1]
{label=\arabic{mylisti}., %1., 2., 3., ...
leftmargin=\parindent,
rightmargin=10pt
}

\begin{mylist}

\item General Description

blah blah blah \\
testing doing this locally i guess if it compiles, it compiles \\
testing again

\item Grammar for the Language

Move Name for each State (Paired together for general buttons/functionality): \\
\begin{table}[h] 
  \begin{tabular}{l | l}
    State & Name  \\
    \hline
    $S$ & Neutral State \\
    $q_{ss}$ & Side Smash \\
    $q_{os1}$ & Overhead Smash I \\
    $q_{os2}$ & Overhead Smash II \\
    $q_{u}$ & Upswing \\
    $q_{bb1}$ & Big Bang I \\
    $q_{bb2}$ & Big Bang II \\
    $q_{bb3}$ & Big Bang III \\
    $q_{bb4}$ & Big Bang IV \\
    $q_{bbf}$ & Big Bang Finisher\\
    $q_{sb1}$ & Big Bang Finisher\\
    $q_{sb2}$ & Big Bang Finisher\\
    $q_{sb3}$ & Big Bang Finisher\\
    $q_{c1}$ & Charge \\
    $q_{c2}$ & Charge \\
    $q_{c3}$ & Charge \\
    $q_{csb}$ & Charge \\
    $q_{cu}$ & Charge \\
    $q_{cbb}$ & Charge \\
    $q_{cfu}$ & Charge \\
    $q_{mc1}$ & Mighty Charge \\
    $q_{mc2}$ & Mighty Charge \\
    $q_{mcu}$ & Mighty Charge \\
    $q_{mcs}$ & Mighty Charge \\
  \end{tabular}
\end{table}
S is the initial variable for the beginning of a combo
%we should use a symbol to denote "hold button". I will use \mu for now
\begin{align*}
&S \rightarrow yq_{os1} \, | \, bq_{bb1} \, | \, r_2q_{c1} \, | \, l_2 \, | \, (ly)(L+Y) \, | \, \lambda \\
%basic side smash/overhead 1 --> overhead 2 --> overhead 3
&q_{ss} \rightarrow yq_{os2} \, | \, b{q_bb1} \, | \, r_2q_{c1} \, | \, (yb)q_{sb1} \, | \, tS\\
&q_{os1} \rightarrow yq_{os2} \, | \, b{q_bb1} \, | \, r_2q_{c1} \, | \, (yb)q_{sb1} \, | \, tS\\
&q_{os2} \rightarrow yq_{u} \, | \, b{q_bb1} \, | \, r_2q_{c1} \, | \, (yb)q_{sb1} \, | \, tS\\
&q_{u} \rightarrow b{q_bb1} \, | \, r_2q_{c1} \, | \,(yb)q_{sb1} \, | \, (r_2yb)q_{mc1} \, | \, tS\\
%big bang
&q_{bb1} \rightarrow bq_{bb2} \, | \, r_2q_{c1} \, | \, tS \\ 
&q_{bb2} \rightarrow bq_{bb3} \, | \, r_2q_{c1} \, | \, tS \\ 
&q_{bb3} \rightarrow bq_{bb4} \, | \, r_2q_{c1} \, | \, tS \\ 
&q_{bb4} \rightarrow bq_{bbf} \, | \, (r_2yb)q_{mc1} \, | \, r_2q_{c1} \, | \, tS \\ 
&q_{bbf} \rightarrow r_2q_{c1} \, | \, (yb)q_{sb1} \, | \, tS \\ 
%sb 
&q_{sb1} \rightarrow tq_{sb2} \, | \, yS? \\ 
&q_{sb2} \rightarrow tq_{sb3} \, | \, yS? \\ 
&q_{sb3} \rightarrow tS? \, | \, yS? \\ 
%charge
&q_{c1} \rightarrow tq_{c2} \, | \,  r_2q_{csb} \, | \, yq_{csb} \, | \, bq_{cs1} \, | \, (l_2r_1)q_{fbe}\\ 
&q_{c2} \rightarrow tq_{c3} \, | \,  r_2q_{csu} \, | \, yq_{csu} \, | \, bq_{cs2} \, | \, (l_2r_1)q_{fbe}\\ 
&q_{c3} \rightarrow r_2q_{cbb} \, | \,  yq_{cbb} \, | \, bq_{cs3} \, | \, (yb)q_{sb1} \, | \, (l_2r_1)q_{fbe}\\ 
%mighty charge (release is omitted)
&(\bar{R_2}+Y+B) \rightarrow (yb)(Y+B) \, | \, y(Y_1) \, | \, b(B_1) \, | \, r_2\bar{(R_2)} \\
%focus mode
&\bar{(L_2)} \rightarrow y(Y_1) \, | \, b(B_1) \, | \, r_2\bar{(R_2)} \, | \, aA \, | \, r_1R_1 \\
%focus strike/charge. when not hitting a wound, shows B | R_2
&\bar{(R_1)} \rightarrow (\bar{R_2}+Y+B) \, | \, bB \, | \, r_2\bar{(R_2)} \\
%charge attack. After 3 time time units, can use y/b to go to mighty charge. Different time units results in seperate combos that can be used. 
&\bar{(R_2)} \rightarrow (\bar{R_2}+Y+B) \, | \, y \, | \, b \, | \, (l_1r_1)\bar{(R_1)} \\ 
&\bar{(R_2)} \rightarrow (\bar{R_2}+Y+B) \, | \, y \, | \, b \, | \, (l_1r_1)\bar{(R_1)} \\
\end{align*}

\item Automaton

\begin{center}

\begin{figure}[ht]
    \centering
    \begin{tikzpicture}
        \node[state, initial](q0){$q_0$};
        \node[state, accepting, right of=q0](q1){$q_1$};
        \draw 
        (q0) edge[above] node{$\lambda,\lambda \rightarrow S$} (q1)
        (q1) edge[right, loop right] node[align=center]
        {$\lambda, S \rightarrow aAS$ \\ 
        $\lambda, S \rightarrow a$ \\
        $\lambda, A \rightarrow SbA$ \\
        $\lambda, A \rightarrow SS$ \\
        $\lambda, A \rightarrow ba$ \\
        $a, a \rightarrow \lambda$ \\ 
        $b, b \rightarrow \lambda$} (q1);
    \end{tikzpicture}
    \caption{Automaton for language}
    \label{fig1:automaton-for-language}
\end{figure}

\end{center}

\newpage
\item Data Structure

\item Testing

\end{mylist}

\end{document}

